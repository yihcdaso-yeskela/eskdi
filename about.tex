\documentclass[a4paper,12pt,russian]{article} %draft
\usepackage[T2A]{fontenc}% Поддержка русских букв



% XeTeX packages
\usepackage[cm-default]{fontspec} % or install lmodern and remove cm-default opt
\usepackage{xunicode} % some extra unicode support
\usepackage{xltxtra} % \XeLaTeX macro


\tolerance=1000
\emergencystretch=0.74cm
\usepackage{indentfirst} %делать отступ в начале параграфа

\usepackage[pdfborder = {0 0 0}]{hyperref} %гиперссылки в документе.

\usepackage[utf8]{inputenc}	% кодировка текста
\usepackage[russian]{babel}	% руссификация по Бабелю
\usepackage{graphics}

\usepackage{svg}
\svgsetup{clean=true,inkscapeformat=pdf}

\usepackage{amsmath, amsfonts} % для расширенных настроек ссылок на формулы
\usepackage{extsizes}	% использование шрифтов большего кегля 

\usepackage{fancyvrb} % Добавляет продвинутые Verbatim и Verb

\usepackage{epsfig} % удобно вставлять рисунки в строку текста
\usepackage[usenames,dvipsnames]{pstricks}
\usepackage{pst-grad} % For gradients
\usepackage{pst-plot} % For axes

\usepackage{graphicx,xcolor}

\usepackage[MakeStamp]{eskdi}
%\usepackage[MakeStamp, SubSectInToc]{eskdi}
%\usepackage[MakeStamp, SubSubSectInToc]{eskdi}
%\usepackage[MakeStamp, ParagraphInToc]{eskdi}
%\usepackage[twoside, MakeStamp, ParagraphInToc]{eskdi}
%\usepackage{eskdi}
%\usepackage[SubSectInToc]{eskdi}
%\usepackage[SubSubSectInToc]{eskdi}
%\usepackage[ParagraphInToc]{eskdi}
%\usepackage[ParagraphInToc, NumIntoSections]{eskdi}
%\usepackage[twoside, ParagraphInToc]{eskdi}
%\usepackage[twoside, MakeEmptyStamp, ParagraphInToc]{eskdi}
%\usepackage[twoside, MakeEmptyStamp]{eskdi}
%\usepackage[MakeEmptyStamp, ParagraphInToc]{eskdi}



\usepackage{array}
\usepackage{tabularx}
\usepackage{supertabular}
\usepackage{longtable} % для создания таблиц, переносящихся на другую страницу
%\usepackage{listingsutf8}%
\usepackage{listings} % для включения листинга кода в приложения. Русский язык глючит.


\lstloadlanguages{bash,[LaTeX]TeX,MetaPost,Clean,Matlab}


\usepackage{textcomp}	% Ввод различных знаков
\usepackage{keystroke} % для отображения символов клавиш
\usepackage{bytefield} %для создания таблиц с битовыми полями
\usepackage{filecontents} %для включения в документ содержимого файлов

\usepackage{tikz} % Пакет для рмсования диаграмм
\usepackage{tikz-timing}[2009/12/09]
\usetikzlibrary{positioning,arrows,automata,plotmarks} %В данном случае нам потребуются positioning и arrows, которые нужны для расположения элементов друг относительно друга и рисования стрелок между ними соответственно.
\usetikzlibrary{shapes,snakes}
\usepackage{schemabloc}

\usepackage{makecell} % Для многострочных ячеек таблицы
\usepackage{colortbl} % Для раскрашивания ячеек в таблицах


%{Arial} {Courier New} 
%{OpenGost Type A TT} {OpenGost Type B TT} % Свободный шрифт. Нет наклонного начертания и дирного начертания
%{GOST type A} % Морально устарел, не свободный, не хватает символа тирэ. Не рекомендуется
%{GOST type B} % Морально устарел, не свободный, не хватает символа тирэ. Не рекомендуется
\gostSetRomanfont{Times New Roman}%
\gostSetSansfont{Times New Roman}%
\gostSetMonofont{Times New Roman}%
\gostSetMainfont{Times New Roman}%
\gostSetStampfont{Arial}%


%\verbatimfont{\fontspec[Scale=1.0]{Arial} \itshape}% % Для замены стиля начертания verbatim и verb
\verbatimfont{\fontspec[Scale=1.0]{Consolas}}% % Для замены стиля начертания verbatim и verb
\newfontfamily{\gostListingfont}[Scale=1.0]{Consolas} % Шрифт для листингов



%\renewcommand{\SetStampfontIt}{\itshape}%



\input commands.tex %Файл включает такие команды как надчёркивание, запрещение переноса ТУ и др.











\setpage % Разметка текста на странице

\begin{document}% Начало самого документа (содержательной части)

\input ./about/title.tex %Здесь информация о названии файла, авторах и т.д.

\maketitle % Поместили Титульный лист


\SetEmptyPage
\SetEvenPage


\makesecondpage


\tableofcontents % Содержание



\input ./about/about_begin.tex

\SetEmptyPage
\input ./about/about_latex.tex

\SetEmptyPage
\input ./about/about_eskdi.tex


\input ./about/tunung_latex.tex
\input ./about/about_tuning_eskdi.tex
\input ./about/sample_manual_eskdi.tex
\input ./about/sample_main_text.tex

\input ./about/about_lic.tex


\input ./about/about_bibliography.tex

\appendix % Приложения. Можно разместить перед списком литературы - тогда список литературы оформится как приложение
\clearpage

\input ./about/about_bibliography.tex

\input ./about/frames_from_2_105_95.tex
\clearpage
\input ./about/sample_subsections_in_appendix.tex
\input ./about/sample_tables.tex
\input ./about/sample_figure.tex
\input ./about/sample_timing.tex
\input ./about/sample_equation.tex
\input ./about/sample_code.tex
\input ./about/sample_keys.tex
\input ./about/sample_state.tex
\input ./about/sample_grafics.tex
\input ./about/sample_schblock.tex

\input ./about/bash/sample_listing_bash.tex
\input ./about/matlab/matlab_generated.tex

\input ./about/sample_systemc.tex
\input ./about/sample_report_power.tex
\input ./about/sample_signature.tex
%\input ./about/sample_listing.tex
\input ./about/sample_big_frames1.tex
\input ./about/sample_big_frames2.tex
\input ./about/sample_label.tex
\input ./about/about_abbreviation.tex

\SetEmptyPage
\input ./about/about_ref.tex

\SetEmptyPage
\regChanges %Вставляем лист регистрации изменений
\sectionmark{Лист замечаний}
\end{document}

