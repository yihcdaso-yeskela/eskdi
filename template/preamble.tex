%Преамбула с набором настроек
\usepackage[T2A]{fontenc}% Поддержка русских букв

% XeTeX packages
\usepackage[cm-default]{fontspec} % or install lmodern and remove cm-default opt
\usepackage{xunicode} % some extra unicode support
\usepackage{xltxtra} % \XeLaTeX macro

\tolerance=1000
\emergencystretch=0.74cm
\usepackage{indentfirst} %делать отступ в начале параграфа

\usepackage[pdfborder = {0 0 0}]{hyperref} %гиперссылки в документе.

\usepackage[utf8]{inputenc}	% кодировка текста
\usepackage[russian]{babel}	% руссификация по Бабелю
\usepackage{graphics}

\usepackage{svg}
\svgsetup{clean=true,inkscapeformat=pdf}

\usepackage{amsmath, amsfonts} % для расширенных настроек ссылок на формулы
\usepackage{extsizes}	% использование шрифтов большего кегля 

\usepackage{fancyvrb} % Добавляет продвинутые Verbatim и Verb

\usepackage{epsfig} % удобно вставлять рисунки в строку текста
\usepackage[usenames,dvipsnames]{pstricks}
\usepackage{pst-grad} % For gradients
\usepackage{pst-plot} % For axes

\usepackage{graphicx,xcolor}


\usepackage{array}
\usepackage{tabularx}
\usepackage{supertabular}
\usepackage{longtable} % для создания таблиц, переносящихся на другую страницу
%\usepackage{listingsutf8}%
\usepackage{listings} % для включения листинга кода в приложения. Русский язык глючит.


\lstloadlanguages{bash,[LaTeX]TeX,MetaPost,Clean,Matlab}


\usepackage{textcomp}	% Ввод различных знаков
\usepackage{keystroke} % для отображения символов клавиш
\usepackage{bytefield} %для создания таблиц с битовыми полями
\usepackage{filecontents} %для включения в документ содержимого файлов

\usepackage{tikz} % Пакет для рмсования диаграмм
\usepackage{tikz-timing}[2009/12/09]
\usetikzlibrary{positioning,arrows,automata,plotmarks} %В данном случае нам потребуются positioning и arrows, которые нужны для расположения элементов друг относительно друга и рисования стрелок между ними соответственно.
\usetikzlibrary{shapes,snakes}
\usepackage{schemabloc}

\usepackage{makecell} % Для многострочных ячеек таблицы
\usepackage{colortbl} % Для раскрашивания ячеек в таблицах

\usepackage{blindtext}